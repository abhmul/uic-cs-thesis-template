\section{Polya's Urn}
\label{stoch:polya}

\section{Borel-Cantelli Lemma}

\section{Differential Equation Method?}

In this section we examine an example studied in \cite{diffeq_2020} about counting the number of connected components of size \(k\) in an Erdos-Renyi graph. First we  construct our probability space. 

We let \(\mathbb{G}_{n}\) be the set of graphs on \([n]\) for some \(n \geq 0\). Denote \(\binom{[n]}{2}\) the set of edges of the complete graph on \([n]\). Each graph in \(G \in \mathbb{G}_{n}\) has the same vertex set, so they can be identified with their edge sets \(E(G) \subset \binom{[n]}{2}\). Thus, we simply refer to a graph in \(\mathbb{G}_{n}\) by its set of edges. 

Let \(n \in \mathbb{N}\), let \(\Omega :=  \binom{[n]}{2}\), and let \(M = \binom{n}{2}!\). Let \(\mathbb{P}: \mathcal{P}(\Omega^{M}) \to [0,1]\) be defined for \(\omega \in \Omega^{M}\) so that
\[\mathbb{P}(\{\omega\}) = \begin{cases} \frac{1}{M!} & \text{ if } \omega \text{ consists of distinct edges} \\ 0 & \text{ otherwise}\end{cases}\]
This probability measure simply assigns uniform probability to all permutations of edges. Let \(\mathcal{F}_{0}= \{\emptyset, \Omega^{M}\}\) and \(\mathcal{F}_{i} = \{S \times \Omega^{M-i} : S \in \mathcal{P}(\Omega^{i})\}\) make up the filtration \(\mathcal{F}_{*} := (\mathcal{F}_{i})_{i=0}^{M}\).

\begin{definition}
    An \textbf{Erdos-Renyi random graph} on \(m \in \{0, \dots, M\}\) edges, \(G_{n, m}: \Omega \to \mathbb{G}_{n}\), is defined on \(\omega \in \Omega\) as
    \[G_{n, m}(\omega) = \{\omega_{1}, \dots, \omega_{m}\}.\]
    That is, it is a random graph sampled uniformly from all graphs on \(n\) vertices with \(m\) edges. 
\end{definition}

We can think of sampling a particular Erdos-Renyi graph as the result of picking an edge \(m\) times without replacement. This is captured by the process defined below:
\begin{definition}
    The \textbf{Erdos-Renyi graph process} is the stochastic process \((G_{n, i}: \Omega \to \mathbb{G}_{n})_{i=0}^{M}\) adapted to \(\mathcal{F}_{*}\).
\end{definition}

We would like to characterize properties about this process as \(n \to \infty\). To do so, we use the following concept:
\begin{definition}
     Let \((X_{n}: \Omega \to \mathbb{R})_{n \geq 1}\) be a stochastic process and let \((x_{n})_{n \geq 1}\subset \mathbb{R}\). Then \(X_{n} = x_{n}\) \textbf{asymptotically almost surely} if there exists \((\epsilon_{n})_{n \geq 1} \subset \mathbb{R}\) so that \(\epsilon_{n} = o(1)\) and
    \[\mathbb{P}((1 - \epsilon_{n})x_{n} \leq X_{n} \leq (1 + \epsilon_{n})x_{n}) \to 1\]
    as \(n \to \infty\). It may be abbreviated a.a.s..
\end{definition}

We would like to prove the following a.a.s. characterization of the number of connected components of a fixed size of an Erdos-Renyi random graph.

\begin{theorem}
    Let \(\kappa \geq 1\) and \(c \in \mathbb{R}^{\geq 0}\). Then a.a.s. the number of connected components of order \(k\), for \(1 \leq k \leq \kappa\), in \(G(n, \lfloor cn \rfloor)\) is
    \[\frac{k^{k-2}}{k!}(2c)^{k-1} e^{-2kc}n\]
\end{theorem}

To prove this theorem, we will apply the \textit{differential equation method}. This method works by approximating a random discrete process (in our case the number of connected components of size \(k\) in our random graph as we sample edges)with a continuous, deterministic, differential equation. We construct the differential equation using the expected change in our tracked random variable. We'll use some process-related heuristic to find a solution to the differential equation. We then show that our error incurred from our continuous approximation is small by bounding it from above and below by a supermartingale and submartingale respectively, then applying the following:

\begin{theorem}[Azuma-Hoeffding's Inequality \cite{diffeq_2020}]
    \label{azuma}
    Let \((Y_{n})_{n \geq 0}\) be a supermartingale and suppose there exists real \(C \geq 0\) \(|Y_{n+1} - Y_{n}| \leq C\) a.s. for \(n \geq 0\). Then, for all \(\lambda \in \mathbb{R}^{\geq 0}\) and \(n \geq 1\),
    \[\mathbb{P}(Y_{n} - Y_{0} \geq  \lambda) \leq \exp\left(-\frac{\lambda^{2}}{2C^{2}n}\right)\]
\end{theorem}

TODO: Proof?

The following will also be useful:

\begin{theorem}[Taylor's Theorem \cite{diffeq_2020}]
    Let \(f: \mathbb{R} \to \mathbb{R}\) be a twice differentiable function on \([a, b]\) for \(a < b \in \mathbb{R}\). Then there exists \(\tau \in (a, b)\) so that
    \[f(b) = f(a) + f'(a)(b-a) + \frac{f''(\tau)}{2}(b-a)^{2}.\]
\end{theorem}

Now we come back to our task of studying the number of fixed size connected components in an Erdos-Renyi graph. Recall we are given \(c \in
\mathbb{R}^{\geq 0}\) and suppose \(n \in \mathbb{N}\). Let \(i \leq \lfloor {cn} \rfloor\) and define \(Y_{k}(i)\) to be the number of components with exactly \(k\) vertices in \(G_{n, i}\).

We would like to show that \(Y_{k}(i) = n y_{k}(t_{i})\) a.a.s. for \(t_{i} := \frac{i}{n}\) and \(y_{k}(t) = \frac{k^{k-2}}{k!}(2c)^{k-1} e^{-2kt}\). In our Erdos-Renyi process, we'll let \(i\) progress to \(\lfloor {cn} \rfloor\) (i.e. let \(t \to c\)) to get the desired result. To prove \(Y_{k}(i) = n y_{k}(t_{i})\) a.a.s., we'll make a (good) guess that \(\epsilon_{n}(t) = n^{-1/3} e^{6 \kappa^{3} t}\) and show
\[n(y_{k}(t_{i}) - \epsilon_{n}(t_{i})) \leq Y_{k}(i) \leq n(y_{k}(t_{i}) + \epsilon_{n}(t_{i}))\]
tends to probability 1.

Let \(\nu_{n} := \inf\limits \{i \geq 0 : Y_{k}(i) \not\in [n(y_{k}(t_{i}) - \epsilon_{n}(t_{i})), n(y_{k}(t_{i}) + \epsilon_{n}(t_{i}))]\}\). Define
\[Y_{k}^{\pm}(i) = Y_{k}(i \wedge \nu_{n}) - n(y_{k}(t_{i \wedge \nu_{n}}) \pm \epsilon_{n}(t_{i \wedge \nu_{n}})) \]
We will show that \(Y_{k}^{+}(i)\) is a supermartingale. Using that, we will use Azuma-Hoeffding (theorem \ref{azuma}) to show that \(\mathbb{P}(\nu_{n} \leq O(n)) = o(1)\). The corresponding computations for \(Y_{k}^{-}\) are symmetric.





% Since, we start out with a graph of no edges, 
% \[Y_{k}(0) = \begin{cases} n & \text{ if }k = 1 \\ 0 & \text{ otherwise} \end{cases}.\]
% Thus, \(y_{1}(0) = 1\) and \(y_{k}(0) = 0\) for \(k \geq 2\). 


TODO:
- Erdos-renyi graph
- Thm 1.2 (and defn of a.a.s.) about number of connected components
- Erdos-renyi process and equivalence at step to G(n, i)
- What is differential equation Method and informally how we will use item
- Useful theorems
    - Taylor's Theorem (no proof)
    - Azuma's inequality (with proof)
        - Chernoff-Cramer bound
        - Hoeffding's Lemma
        - Azuma-Hoeffding inequality
    - Cayley's formula (no proof)
- stopping time for bad event
- proof it is supermartingale
- proof guessed solution solves differential equation
- application of azuma-hoeffding
