\summary

We will work to first build up the basic theory of Stochastic Processes and Martingales. We will establish conditional expectation, but take most of the basic results and relevant measure theoretic results for granted. From there, we will turn to martingales and start with studying "strategies" on martingales and how they help us prove convergence. From there we will look into decomposition of processes in terms of martingales and a handful of important examples.

With an understanding of martingales, we will then turn to Markov Processes. [[TODO]]

Brownian Motion is both a martingale and markov process, so our prior work will help elucidate some of its properties. Unlike in the previous sections, however, we will not study brownian motion in isolation. Instead, we will examine its deep relationship with a deterministic object, the heat equation.

Finally, we will put together our work from the previous 3 chapters towards a study of Branching Brownian Motion. Branching Brownian motion involves Brownian motions with lifetimes determined by a standard exponential distribution. Upon death, a Brownian motion will split into two Brownian motions, and so on. Like with regular Brownian motion, we will find an interesting connection with diffusion partial differential equations. With a basic understanding of this connection in hand, we will use it to study the distribution of the maximal point of a standard branching Brownian motion.