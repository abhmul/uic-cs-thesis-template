\section{Conditional Expectation}
\label{stoch:cexpe}

Suppose we are given a probability space \((\Omega, \mathcal{B}, \mathbb{P})\) and a random variable \(X \in L^{1}(\mathcal{B})\). We would like to describe the operation of "viewing" this random variable from a sub-sigma algebra \(\mathcal{G} \subset \mathcal{B}\). We call this operation \textit{conditional expectation} and define it as follows:

\begin{definition}
   The conditional expectation of \(X\) given \(\mathcal{G}\), denoted \(\mathbb{E}({X}|{\mathcal{G}})\), is a random variable with the following properties
    \begin{enumerate}
        \item \(\mathbb{E}({X}|{\mathcal{G}}) \in \mathcal{G}\)
        \item \(\int\limits_{A} X d \mathbb{P} = \int\limits_{A} \mathbb{E}({X}|{\mathcal{G}}) d \mathbb{P}\) for all \(A \in \mathcal{G}\).
    \end{enumerate}
\end{definition}

The existence and \(\mathbb{P}\)-a.s. uniqueness of a conditional expectation follows from the following measure-theoretic theorem.

\begin{theorem}[Radon-Nikodym Theorem]
    \label{rn-theorem}
    Let \(X, \mathcal{M}\) be a measure space with sigma-finite nonnegative measure \(\mu\) and sigma-finite signed measure \(\nu\) so that \(\nu << \mu\). Then there exists \(\mu\)-a.e. unique \(f \in L^{1}(\mathcal{M})\) so that for all \(A \in \mathcal{M}\),
    \[\int\limits_{A}f d \mu = \nu(A).\]
    We denote \(\frac{d \nu}{d \mu} := f\).
\end{theorem}

Note that \(\nu(A) = \int\limits_{A} X d \mathbb{P}\) is a signed measure on \((\Omega, \mathcal{B})\) and \(\mathbb{P}\) is a finite measure. Restricting \(\nu\) to \(\mathcal{G}\) does not change this fact, and applying \ref{rn-theorem} to \(\nu{\big|}_{\mathcal{G}}\) gives us our \(\mathbb{P}\)-a.s. unique conditional expectation
\[\mathbb{E}({X}|{\mathcal{G}}) = \frac{d \nu{\big|}_{\mathcal{G}}}{d \mathbb{P}{\big|}_{\mathcal{G}}} \]
in \(L^{1}(\mathcal{G})\).

